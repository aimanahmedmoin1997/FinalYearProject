    
\section{Introduction}

This chapter sheds light on the overarching theme of this project - The Third Space - by looking at the current work done in this particular domain by reviewing academic articles and papers.

As introduced in above sections, this project aims to target two sub domains, i.e. Rights and Employment Opportunities of marginalized communities and market condition(s) of small-scale Arts and Crafts businesses. Therefore, this Literature Review, while taking into account the current state(s) and other similar work that has been done in the area of art and crafts and employment opportunities for marginalized communities, also establishes the novelty of our work by highlighting the differences between the existing work and our work.

This Review will comprise of statistics taken from the cited literature as well as an analysis of primary text. We'll first review the literary sources catering to the domain of the plight of marginalized communities. Followed by this will be the review of the literary sources related to the Pakistani handicraft industry which will then lead to our discussion of rising e-commerce industry in Pakistan.

There are a large number of studies about the trans-gender communities and much is known and written about 'hijras' (a trans-person) in India, however, very little is documented about them in Pakistan (Jami, 2005). It is also to note that the focus of this research is the sad state of these communities with regards to their rights and employment opportunities, therefore this research will use the umbrella term `trans-gender' instead of going into the depth of multiple terms ('Zanana', 'Khuwaja Sira', 'Mukhannas', 'Hijra') used within South Asian culture (Winter, 2002). 

\section{Conceptual Frame Work}

\section{Theoretical Frame Work}

\section{Emperical Review of Previous Work(s)}

In this section, we'll look at the research work done already by local and international researchers in the three domains our project is related with. This section will take on these three domains in chronological order and discuss the various researchers' works  and the results they arrived while also putting them in context.

\subsection{Plight of the marginalized communities}

There is limited literature based on primary data of trans-gender persons in Pakistan (Nazir \& Yasir, 2016). Just the factor that we have little to no documentation for an entire community in relation to their population, literacy rate, employment rate, etc. is telling enough as to how our society has devoid them of any rights. This is the particular reason why the author(s) of this report chose to work in this domain. The need for this cause to be taken up by scientists and engineers today is far more important so that we can find solutions with the aid of technology and create a better world for all of us. 

In a research done by Jami, H. in 2005 with the name \textit {Condition and status of hijras (trans-gender, transvestites etc) in Pakistan}, she concludes that,“ `We hate some people but we do not know them and we do not want to know them because we hate them', this dictum stands valid in our attitude towards hijras." This research concludes that the discriminating attitude is in effect by and large in our society against the personalities of trans-gender communities. It can also be drawn from the study that not only Pakistanis have the instilled bias against the trans-gender persons and people detest the idea of having a trans-gender in a family, but they are also considered responsible for sex business and homosexuality (Jami, 2005) both of which in a society as religiously conservative as Pakistan's, are considered blatant sins and is also one of the root causes of violence against the member of these communities. This study also confirms that women tend to have positive attitude towards trans-gender persons compared to men (who are also the performers and executers of violence on the members of this already marginalized community). 

Another research done by Tabassum, S. \& Jamil, S. in (2014) by the title of \textit {Plight of marginalized: Educational issues of trans-gender community in Pakistan} also affirms the same as Zubaida (a trans-gender woman) states “Brothers and fathers never bother to think about our (trans-gender person) situation perhaps they thanked to God if we left the home”. Tabassum \& Jamil also concludes in their study that there is a greater desire (of almost 94\% of the sample population from the community) to acquire education and it's their belief that education is a necessary tool for their upwards mobility. However, it was also gathered that some of them who tried to get education faced lot of problems in terms of their enrollment in schools, group selection in the class rooms and in answering the unknown questions of the fellows (Tabassum, S., \& Jamil, S., 2014). 

The findings from the research conducted by Nazir, N., \& Yasir, A. in 2016 in the northern province of Pakistan by the title \textit {Education, Employability and Shift of Occupation of Trans-gender in Pakistan: A Case Study of Khyber Pakhtunkhwa.} showed that trans-gender people have experienced unemployment twice the rate of the population as a whole. 97\% of the surveyed population was facing mistreatment on the job. Out of total 47\% faced an adverse job outcome, including job refusal, or being fired or denied promotion. 26\% lost their job because of being trans-gender. 15\% of the sampled respondents lived in poverty which was double the rate of the general population (Nazir \& Yasir, 2016). This study also talks about the stigma attached to the families of trans-gender persons and how society continuously taunts them till they disown their child which leads them to being deprived of normal life, education and later on earning means that are considered honorable in the society (Nazir \& Yasir, 2016). It is also to note that this report also talks about the landmark judgment by the Supreme Court of Pakistan which directed both federal and provincial governments to ensure the rights of education, employment and inheritance (Nazir \& Yasir, 2016). Yet, the measures taken by the state in accordance with the court's ruling are not concrete enough to provide the support they need.

Contextualizing the above findings of all of these reports, we can easily draw the conclusion that the community under consideration is one of the many marginalized communities in our society and there are various factors acting behind this sad state of affairs. Therefore, the need for such a project which could create employment opportunities for them is due well in time.

\subsection{Plight handicraft industry}

\subsection{Rising E-commerce in Pakistan}


\section{Overview of Literature}

\section{References}




















Of course, we take inspiration from \cite{einstein} but wish the work was typeset in \LaTeX \cite{knuthwebsite}, e.g. by taking help from \cite{latexcompanion}.