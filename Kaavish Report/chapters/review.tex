

\section{Introduction}

This chapter sheds light on the overarcial theme of this project - The Third Space - by looking at the current work done in this particular domain by reviewing academic articles and papers.

As introduced in above sections, this project aims to target two sub domains, i.e. Rights and Employment Opportunities of marginalized communities and market condition(s) of small-scale Arts and Crafts bussinesses. Therefore, this Literature Review, while taking into account the current state(s) and other similar work that has been done in the area of art and crafts and employment opportunities for marginalized communities, also establishes the novelty of our work by highlighting the differences between the existing work and our work.

This Review will comprise of statistics taken from the cited literature as well as an analysis of primary text. We'll first review the literary sources catering to the domain of the plight of marginalized communities. Followed by this will be the review of the literary sources related to the Pakistani handicraft industry which will then lead to our discussion of rising e-commerce industry in Pakistan.

There are a large number of studies about the transgender communities and much is known and written about 'hijras' (a transperson) in India, however, very little is documented about them in Pakistan (Jami, 2005). It is also to note that the focus of this research is the sad state of these communities with regards to their rights and employment opportunities, therefore this research will use the umbrella term 'transgender' instead of going into the depth of multiple terms ('Zanana', 'Khuwaja Sira', 'Mukhannas', 'Hijra') used within South Asian culture (Winter, 2002). 

\section{Overview of Previous Work}

\subsection{Plight of marginalized communities}

\subsection{Plight handicraft industry}

\subsection{Rising E-commerce in Pakistan}




























Of course, we take inspiration from \cite{einstein} but wish the work was typeset in \LaTeX \cite{knuthwebsite}, e.g. by taking help from \cite{latexcompanion}.