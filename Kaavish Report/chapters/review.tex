
\section{Introduction}

This chapter sheds light on the overarcial theme of this project - The Third Space - by looking at the current work done in this particular domain by reviewing academic articles and papers.

As introduced in above sections, this project aims to target two sub domains, i.e. Rights and Employment Opportunities of marginalized communities and market condition(s) of small-scale Arts and Crafts bussinesses. Therefore, this Literature Review, while taking into account the current state(s) and other similar work that has been done in the area of art and crafts and employment opportunities for marginalized communities, also establishes the novelty of our work by highlighting the differences between the existing work and our work.

This Review will comprise of statistics taken from the cited literature as well as an analysis of primary text. We'll first review the literary sources catering to the domain of the plight of marginalized communities. Followed by this will be the review of the literary sources related to the Pakistani handicraft industry which will then lead to our discussion of rising e-commerce industry in Pakistan.

There are a large number of studies about the transgender communities and much is known and written about 'hijras' (a transperson) in India, however, very little is documented about them in Pakistan (Jami, 2005). It is also to note that the focus of this research is the sad state of these communities with regards to their rights and employment opportunities, therefore this research will use the umbrella term 'transgender' instead of going into the depth of multiple terms ('Zanana', 'Khuwaja Sira', 'Mukhannas', 'Hijra') used within South Asian culture (Winter, 2002). 

\section{Conceptual Frame Work}

\section{Theoretical Frame Work}

\section{Emperical Review of Previous Work(s)}

Now, we'll talk about the research work done already by local and international researchers in the three domains our work is related with. This section will take on these three domains in chronoligal order and discuss the various researchers' work  and the results they arrived while also putting them in context.

\subsection{Plight of the marginalized communities}

There is limited literature based on primary data of transgender persons in Pakistan (Nazir \& Yasir, 2016). Just the factor that we have little to no documentation for an entire community in relation to their population, literacy rate, employment rate, etc. is telling enough as to how our society has devoid them of any rights. This is the particular reason why the author(s) of this report chose to work in this domain. The need for this cause to be taken up by scientists and engineers today is far more important so that we can find solutions with the aid of technology and create a better world for all of us. In a research done by Jami (2005) with the name \textit {Condition and status of hijras (transgender, transvestites etc) in Pakistan}, she concludes that, "“We hate some people but we do not know them and we do not want to know them because we hate them", this dictum stands valid in our attitude towards hijras." This reserch concludes that the discrimating attitude is in effect by and large in our society against the personalities of transgender communities. It can also be drawn from the study that not only Pakistanis have the instilled bias against the transgender persons and people detest the idea of having a transgender in a family, but they are also considered respnosible for sex business and homosexuality (Jami, 2005) both of which in a society as religiously conservative as Pakistan's, are considered blatant sins and is also one of the root causes of violence against the member of these communities. This study also confirms that women tend to have positive attitude towards transgender personscompared to men (who are also the performers and executers of violence on the members of this already marginalized community). Another reserch done by Tabassum, S. \& Jamil, S. in (2014) by the title of \textit {Plight of marginalized: Educational issues of transgender community in Pakistan} also affirms the same as Zubaida (a transgender woman) states “Brothers and fathers never bother to think about our (transgender person) situation perhaps they thanked to God if we left the home”. 



\subsection{Plight handicraft industry}

\subsection{Rising E-commerce in Pakistan}


\section{Overview of Literature}

\section{References}




















Of course, we take inspiration from \cite{einstein} but wish the work was typeset in \LaTeX \cite{knuthwebsite}, e.g. by taking help from \cite{latexcompanion}.