\section{Problem Statement}

Lack of a single web-based platform, with a simple and more user-friendly web app –
feasible to use for anyone, which facilitates the indigenous artists and skilled workers belonging to
marginalized communities (specifically the transgender community) in showcasing their talents and
connecting with the larger audience.
“Through a cross-sectional study design, a sample of 189 transgender community living in twin cities of
Rawalpindi and Islamabad was selected using snowball sampling technique.
The mean age of our study population was 29 ± 7.88 years. Around 94.2% (n=178) were unemployed.
Majority (65.6%) were having monthly income of less than 10000; much less than the government's new
laid-down minimum pay scale of Rs. 14,000/month (announced in 2016 budget).” [1]
Transgender are one of the neglected communities of Pakistani Society. Even after the passing laws on
Transgender Rights, the amount of jobs provided to them are not enough to cater to their daily needs
and earnings. Those who are provided with jobs are usually subjected to public harassment even after
the existence of law against it. In addition to that, they lack educational and vocational training to
support themselves. [2]

As part of our entrepreneurship course, we worked closely with the transgender
community to learn their problems and identify the skillsets to devise a platform which can help them
earn a respectable and easy living, accordingly.
Later in the course, we opted for “food resources” and went forward with the idea of the trans-cuisine.
As a learning from our course (through field work and online reviews/research), we reiterated our 
thought process and after much consideration and guidance from the mentors and the ‘Guru(s)’ (the
term used within the South-Asian Transgender community for their leaders) of the community itself, we
decided to design a platform which can highlight most of their skills and give them a gateway to learn
and achieve more. Thus, came in the idea of “The Third Space”.


\section{Proposed Solution}

This is an e-commerce platform which will work like a virtual mall, each vendor/employee having their
own shop (profile). To simplify process for them, we decided to implement a 3-step application which
will help the seller to upload their products image on the website.
In line with the simplification process, we will be incorporating the non-conventional and easier
payment methods of cash transactions, i.e. easyPaisa and JazzCash (both are local mobile wallets that
allow the users to perform cash transactions using their cellphones only. It is also to consider that
easyPaisa has over 70000+ shops across Pakistan which helps for a wide coverage).
Additionally, we will also incorporate ‘Urdu’ as a mode of communication alongside English. 
A detailed description of each module of the system is presented later in Chapter ~\ref{chap:intro}.

\section{Intended User}

This section outlines the target users of this system. The different types of users in our user base and their interaction with the system are described briefly.

\section{Key Challenges}

This section mentions the key challenges that we foresee in this project and possible ways to address them.