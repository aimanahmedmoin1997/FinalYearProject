\section{Problem Statement}

Lack of a single web-based platform, with a simple and more user-friendly web app –
feasible to use for anyone, which facilitates the indigenous artists and skilled workers belonging to
marginalized communities (specifically the transgender community) in showcasing their talents and
connecting with the larger audience.
“Through a cross-sectional study design, a sample of 189 transgender community living in twin cities of
Rawalpindi and Islamabad was selected using snowball sampling technique.
The mean age of our study population was 29 ± 7.88 years. Around 94.2% (n=178) were unemployed.
Majority (65.6%) were having monthly income of less than 10000; much less than the government's new
laid-down minimum pay scale of Rs. 14,000/month (announced in 2016 budget).” [1]
Transgender are one of the neglected communities of Pakistani Society. Even after the passing laws on
Transgender Rights, the amount of jobs provided to them are not enough to cater to their daily needs
and earnings. Those who are provided with jobs are usually subjected to public harassment even after
the existence of law against it. In addition to that, they lack educational and vocational training to
support themselves. [2]

As part of our entrepreneurship course, we worked closely with the transgender
community to learn their problems and identify the skillsets to devise a platform which can help them
earn a respectable and easy living, accordingly.
Later in the course, we opted for “food resources” and went forward with the idea of the trans-cuisine.
As a learning from our course (through field work and online reviews/research), we reiterated our 
thought process and after much consideration and guidance from the mentors and the ‘Guru(s)’ (the
term used within the South-Asian Transgender community for their leaders) of the community itself, we
decided to design a platform which can highlight most of their skills and give them a gateway to learn
and achieve more. Thus, came in the idea of “The Third Space”.


\section{Proposed Solution}

This is an e-commerce platform which will work like a virtual mall, each vendor/employee having their
own shop (profile). To simplify process for them, we decided to implement a 3-step application which
will help the seller to upload their products image on the website.
In line with the simplification process, we will be incorporating the non-conventional and easier
payment methods of cash transactions, i.e. easyPaisa and JazzCash (both are local mobile wallets that
allow the users to perform cash transactions using their cellphones only. It is also to consider that
easyPaisa has over 70000+ shops across Pakistan which helps for a wide coverage).
Additionally, we will also incorporate ‘Urdu’ as a mode of communication alongside English. 
A detailed description of each module of the system is presented later in Chapter ~\ref{chap:intro}.

\section{Intended User}

The prime user for this system will be the vendor(s) of this e-commerce website who will be opening their shops online to sell their products. We intend to incorporate the marginalized communities by collaborating with various Non Govenrment Organizations in order to take on board artisans and skilled workers from diverse communities. Moreover, the other user for this system will be the buyer(s) who will be using this platform to buy different articles and artifacts. The primary way of interaction between these two users will be our system, which will connect the buyer and the seller through web and app based platform. The seller (vendor) will be using an additional feature of mobile application which will provide them an easy channel to upload their products on their e-shop(s). The details pertaining to the interaction of buyers and sellers with our system are provided in the later chapters.

\section{Key Challenges}

The key challenges we foresee as of now includes the search for free APIs as all APIs we need to use are paid and we being students, lack the monitory resources required. In order to host our website, we also require funds. In addition to such technical challenges, we also realize we will also face challenges in finding enough content to make available on our website for it to become a functioning e-commerce space. For this we have onboard with us a few NGOs who are working with transgender communities that will help us in making resources available for our marketplace. Combatting prejudices within our society when it comes to interactions with transgender people will also be another challenge and to counter this, we not only will share their products with our audience, but also their skillsets and stories, giving them a greater scope and recognition for finding employment opportunities. 