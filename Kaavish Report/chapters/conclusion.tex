The theme surrounding our project was to explore the intersectionality between the rights and employment opportunies of marginalzied communities (especially transgenders), the handicraft industry and the rising e-commerce industry - all in a local Pakistani context. The final product will aim to be a hopeful endeavour in that regard.\\
As a team with relatively low development experience, we have learned and applying new web development programming techniques and concepts into our project while maintaing our envisioned goals and outcomes. As part of building our web application, we used knowledge of HTML, CSS and Javascript and were introduced to the powerful React framework, its component-based structure helping to create efficient and reusable code. Along with that, we used NodeJS (built on Chrome's Javascript V8 Engine) that helped us run and create real time network applications. This further utilized MongoDB, the first NOSQL database client that we had used. In addition, we were acclimatizing to the process of looking up new libraries and tools that would supplement our work, reading into documentations and learning to integrate them into our project for e.g Bootstrap, JWT, Express, Multer and so on. For our mobile application we also another framework by React, React Native - accomodated by the Expo-CLI toolchain used to wrap up the complexity of React Native. We combined that with Google's Cloud Database Firebase for easier access and authentication of user (with real-time update).\\
As beginners into web development, it is not only the technological frameworks, but also the pre-implementation method that was extremely beneficial and important to our work. Laying out the groundwork with a clear identification of the problem to be solved, backed up by a comprehensive literature review helped us gain an understanding of the main objectives we had. In our bid to pursue these objectives, the laying out of our requirements and pre-planning and envisioning the interfaces, use cases, the designs, the database, the different modules was a comprehensive learning experience and a great insight into how the foundation work of a software development process is extremely essential to the task. In light of that groundwork that was laid, there were some apparent changes and modifications that came about due to circumstances and due to the process of learning, trying, failing and getting back up.\\
With that we must also acknowledge that our objectives are part of a bigger process and one that will require constant effort. In terms of the future of the project, we are looking to invite as many indigenous and marginalzied artisans as we can to join The Third Space and use the platform to provide a stronger and reliable e-commerce solution, and to attract customers to a portal that provides services which can contribute to more healthier and responsible e-commerce habit.